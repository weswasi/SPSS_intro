% Options for packages loaded elsewhere
\PassOptionsToPackage{unicode}{hyperref}
\PassOptionsToPackage{hyphens}{url}
\PassOptionsToPackage{dvipsnames,svgnames,x11names}{xcolor}
%
\documentclass[
]{book}
\usepackage{amsmath,amssymb}
\usepackage{lmodern}
\usepackage{iftex}
\ifPDFTeX
  \usepackage[T1]{fontenc}
  \usepackage[utf8]{inputenc}
  \usepackage{textcomp} % provide euro and other symbols
\else % if luatex or xetex
  \usepackage{unicode-math}
  \defaultfontfeatures{Scale=MatchLowercase}
  \defaultfontfeatures[\rmfamily]{Ligatures=TeX,Scale=1}
\fi
% Use upquote if available, for straight quotes in verbatim environments
\IfFileExists{upquote.sty}{\usepackage{upquote}}{}
\IfFileExists{microtype.sty}{% use microtype if available
  \usepackage[]{microtype}
  \UseMicrotypeSet[protrusion]{basicmath} % disable protrusion for tt fonts
}{}
\makeatletter
\@ifundefined{KOMAClassName}{% if non-KOMA class
  \IfFileExists{parskip.sty}{%
    \usepackage{parskip}
  }{% else
    \setlength{\parindent}{0pt}
    \setlength{\parskip}{6pt plus 2pt minus 1pt}}
}{% if KOMA class
  \KOMAoptions{parskip=half}}
\makeatother
\usepackage{xcolor}
\usepackage{color}
\usepackage{fancyvrb}
\newcommand{\VerbBar}{|}
\newcommand{\VERB}{\Verb[commandchars=\\\{\}]}
\DefineVerbatimEnvironment{Highlighting}{Verbatim}{commandchars=\\\{\}}
% Add ',fontsize=\small' for more characters per line
\usepackage{framed}
\definecolor{shadecolor}{RGB}{248,248,248}
\newenvironment{Shaded}{\begin{snugshade}}{\end{snugshade}}
\newcommand{\AlertTok}[1]{\textcolor[rgb]{0.94,0.16,0.16}{#1}}
\newcommand{\AnnotationTok}[1]{\textcolor[rgb]{0.56,0.35,0.01}{\textbf{\textit{#1}}}}
\newcommand{\AttributeTok}[1]{\textcolor[rgb]{0.77,0.63,0.00}{#1}}
\newcommand{\BaseNTok}[1]{\textcolor[rgb]{0.00,0.00,0.81}{#1}}
\newcommand{\BuiltInTok}[1]{#1}
\newcommand{\CharTok}[1]{\textcolor[rgb]{0.31,0.60,0.02}{#1}}
\newcommand{\CommentTok}[1]{\textcolor[rgb]{0.56,0.35,0.01}{\textit{#1}}}
\newcommand{\CommentVarTok}[1]{\textcolor[rgb]{0.56,0.35,0.01}{\textbf{\textit{#1}}}}
\newcommand{\ConstantTok}[1]{\textcolor[rgb]{0.00,0.00,0.00}{#1}}
\newcommand{\ControlFlowTok}[1]{\textcolor[rgb]{0.13,0.29,0.53}{\textbf{#1}}}
\newcommand{\DataTypeTok}[1]{\textcolor[rgb]{0.13,0.29,0.53}{#1}}
\newcommand{\DecValTok}[1]{\textcolor[rgb]{0.00,0.00,0.81}{#1}}
\newcommand{\DocumentationTok}[1]{\textcolor[rgb]{0.56,0.35,0.01}{\textbf{\textit{#1}}}}
\newcommand{\ErrorTok}[1]{\textcolor[rgb]{0.64,0.00,0.00}{\textbf{#1}}}
\newcommand{\ExtensionTok}[1]{#1}
\newcommand{\FloatTok}[1]{\textcolor[rgb]{0.00,0.00,0.81}{#1}}
\newcommand{\FunctionTok}[1]{\textcolor[rgb]{0.00,0.00,0.00}{#1}}
\newcommand{\ImportTok}[1]{#1}
\newcommand{\InformationTok}[1]{\textcolor[rgb]{0.56,0.35,0.01}{\textbf{\textit{#1}}}}
\newcommand{\KeywordTok}[1]{\textcolor[rgb]{0.13,0.29,0.53}{\textbf{#1}}}
\newcommand{\NormalTok}[1]{#1}
\newcommand{\OperatorTok}[1]{\textcolor[rgb]{0.81,0.36,0.00}{\textbf{#1}}}
\newcommand{\OtherTok}[1]{\textcolor[rgb]{0.56,0.35,0.01}{#1}}
\newcommand{\PreprocessorTok}[1]{\textcolor[rgb]{0.56,0.35,0.01}{\textit{#1}}}
\newcommand{\RegionMarkerTok}[1]{#1}
\newcommand{\SpecialCharTok}[1]{\textcolor[rgb]{0.00,0.00,0.00}{#1}}
\newcommand{\SpecialStringTok}[1]{\textcolor[rgb]{0.31,0.60,0.02}{#1}}
\newcommand{\StringTok}[1]{\textcolor[rgb]{0.31,0.60,0.02}{#1}}
\newcommand{\VariableTok}[1]{\textcolor[rgb]{0.00,0.00,0.00}{#1}}
\newcommand{\VerbatimStringTok}[1]{\textcolor[rgb]{0.31,0.60,0.02}{#1}}
\newcommand{\WarningTok}[1]{\textcolor[rgb]{0.56,0.35,0.01}{\textbf{\textit{#1}}}}
\usepackage{longtable,booktabs,array}
\usepackage{calc} % for calculating minipage widths
% Correct order of tables after \paragraph or \subparagraph
\usepackage{etoolbox}
\makeatletter
\patchcmd\longtable{\par}{\if@noskipsec\mbox{}\fi\par}{}{}
\makeatother
% Allow footnotes in longtable head/foot
\IfFileExists{footnotehyper.sty}{\usepackage{footnotehyper}}{\usepackage{footnote}}
\makesavenoteenv{longtable}
\usepackage{graphicx}
\makeatletter
\def\maxwidth{\ifdim\Gin@nat@width>\linewidth\linewidth\else\Gin@nat@width\fi}
\def\maxheight{\ifdim\Gin@nat@height>\textheight\textheight\else\Gin@nat@height\fi}
\makeatother
% Scale images if necessary, so that they will not overflow the page
% margins by default, and it is still possible to overwrite the defaults
% using explicit options in \includegraphics[width, height, ...]{}
\setkeys{Gin}{width=\maxwidth,height=\maxheight,keepaspectratio}
% Set default figure placement to htbp
\makeatletter
\def\fps@figure{htbp}
\makeatother
\setlength{\emergencystretch}{3em} % prevent overfull lines
\providecommand{\tightlist}{%
  \setlength{\itemsep}{0pt}\setlength{\parskip}{0pt}}
\setcounter{secnumdepth}{5}
\usepackage{booktabs}
\ifLuaTeX
  \usepackage{selnolig}  % disable illegal ligatures
\fi
\usepackage[]{natbib}
\bibliographystyle{plainnat}
\IfFileExists{bookmark.sty}{\usepackage{bookmark}}{\usepackage{hyperref}}
\IfFileExists{xurl.sty}{\usepackage{xurl}}{} % add URL line breaks if available
\urlstyle{same} % disable monospaced font for URLs
\hypersetup{
  pdftitle={SPSS introduktion},
  pdfauthor={Enes Al Weswasi, Olof Bäckman, Anders Nilsson och Fredrik Sivertsson},
  colorlinks=true,
  linkcolor={Maroon},
  filecolor={Maroon},
  citecolor={Blue},
  urlcolor={Blue},
  pdfcreator={LaTeX via pandoc}}

\title{SPSS introduktion}
\author{Enes Al Weswasi, Olof Bäckman, Anders Nilsson och Fredrik Sivertsson}
\date{2022-08-03}

\begin{document}
\maketitle

{
\hypersetup{linkcolor=}
\setcounter{tocdepth}{1}
\tableofcontents
}
\hypertarget{om-spss-introduktion}{%
\chapter*{Om SPSS Introduktion}\label{om-spss-introduktion}}
\addcontentsline{toc}{chapter}{Om SPSS Introduktion}

LÅNGT IFRÅN FÄRDIG - UNDER KONSTRUKTION

Denna SPSS-introduktion är avsedd för dig som läser på Kriminologiska institutionen vid Stockholms universitet. På vänster sida har nu de olika teman som guiden består av. Samtliga guider innehåller instruktioner i textformat men i slutet av varje guide finns även videoinstruktioner med alla de steg som beskrivits i texten.

Första temat rymmer en genomgång hur man installerar SPSS och öppnar ett dataset. Därefter följer en genomgång av SPSS olika fönster. Nästa tema ger grundläggande tillvägagångssätt för bearbetning av data. Därefter kommer första temat för dataanalys vilket består av en genomgång om hur man tar fram deskriptiv statistik. Andra temat av dataanalys består av en genomgång av sambandsanalys av två variabler. I det sista temat om datanalys preseneras en rad analysmetoder för bivariata och multivariata samband. Guiden avslutas med en genomgång hur man bearbetar figurer och tabeller så att de blir presentationsdugliga samt en presentation av de dataset några av de dataset som finns att tillgå.

SPSS introduktion ger en guide till att arbeta med SPSS Version 28 vilket är den senaste versionen och som vi rekommenderar att ni använder.

\hypertarget{part-komma-iguxe5ng-med-spss}{%
\part*{Komma igång med SPSS}\label{part-komma-iguxe5ng-med-spss}}
\addcontentsline{toc}{part}{Komma igång med SPSS}

\hypertarget{installera-och-starta-spss}{%
\chapter{Installera och starta SPSS}\label{installera-och-starta-spss}}

\hypertarget{ladda-ner-spss}{%
\section{Ladda ner SPSS}\label{ladda-ner-spss}}

Nedanstående instruktioner är hämtade från Stockholms universitets sida om SPSS. Om någon länk ej fungerar så pröva att besök SU:s SPSS-sida. Där hittar ni även information om hur man avinstallerar SPSS eller förnyar licensnyckeln.

\begin{enumerate}
\def\labelenumi{\arabic{enumi}.}
\item
  Börja med att klicka in er till sidan för installationsfiler för SPSS.
\item
  Välj vilket operativsystem du vill installera SPSS på. SPSS finns till Windows, Mac, och Linux. Observera att SPSS inte finns till Chromebook.
\item
  Olika versioner av SPSS finns tillgängliga men vi rekommenderar version 28 eftersom ``SPSS Introduktion'' är anpassad efter denna version. Välj vilken version du vill installera (samt om du kör x32 eller x64 på Windows) \textgreater{} Klicka på .exe- .pkg-, respektive .dmg-filen \textgreater{} Välj Spara fil/Save File.
\item
  För SPSS på Windows, finns ZIPade installationsfiler. Spara ner den ZIPade filen på skrivbordet \textgreater{} högerklicka på mappen och välj Extract All \textgreater{} välj Extract \textgreater{} dubbelklicka på setup.exe \textgreater{} välj Run.
\end{enumerate}

\hypertarget{installera-spss}{%
\section{Installera SPSS}\label{installera-spss}}

\begin{enumerate}
\def\labelenumi{\arabic{enumi}.}
\tightlist
\item
  Gå igenom installations-wizarden.
\end{enumerate}

\begin{itemize}
\item
  Om du får ett val mellan Authorized user license och concurrent user license \textgreater{} välj Authorized user license.
\item
  Om du ska ange Organization \textgreater{} skriv Stockholms universitet.
\end{itemize}

\begin{enumerate}
\def\labelenumi{\arabic{enumi}.}
\setcounter{enumi}{1}
\item
  När du har gått igenom wizarden \textgreater{} välj Install (detta kan ta några minuter).
\item
  När installationen är genomförd \textgreater{} välj License Product. Du kommer att se en lista på de applikationer som är temporärt licensierade i 14 dagar. Klicka på Next \textgreater{} se till att alternativet Authorized User License är valt \textgreater{} välj Next.
\item
  För att få åtkomst till aktuella behörighetskoder, klickar du här. Klistra in koden för den version du har valt att licensiera \textgreater{} välj Next. Du ser nu en progressruta och denna avslutas med ``\textbf{\emph{End of Transaction}}'' Successfully processed all codes \textgreater{} välj Next. Du ser nu en ruta som bekräftar de licensierade modulerna. Klicka på Finish för att avsluta licensieringen.
  Verifiera
\item
  Verifiera installationen genom att kontrollera att det går bra att starta programmet.
\end{enumerate}

\hypertarget{starta-spss}{%
\section{Starta SPSS}\label{starta-spss}}

Öppna SPSS genom att dubbelklicka på ikonen på skrivbordet eller gå via starmenyn När du öppnar SPSS kommer du att, som på bilden nedan, se en datamatris -- ett rutnät bestående av rader och kolumner.

Raderna i denna matris motsvarar observationsenheter (t.ex. individer) medan kolumner motsvarar variabler (t.ex. frågor i en surveyundersökning).

Om du vill öppna ett befintligt dataset väljer du:

\begin{enumerate}
\def\labelenumi{\arabic{enumi}.}
\tightlist
\item
  File \textgreater{} Open \textgreater{} Data
\item
  Hitta datafilen i katalogen och välj ''Open''
\item
  I exemplet nedan öppnas filen Skol05.sav från datorns skrivbord
\end{enumerate}

\hypertarget{viktiga-fuxf6nser-i-spss}{%
\chapter{Viktiga fönser i SPSS}\label{viktiga-fuxf6nser-i-spss}}

SPSS innehåller flera olika fönster med skilda
funktioner.

\hypertarget{datamatrisfuxf6nstret-data-view}{%
\section{Datamatrisfönstret (Data view)}\label{datamatrisfuxf6nstret-data-view}}

Detta fönster redovisar variablerna (t.ex. en
fråga i en enkäteundersökning) kolumnvis
och observationsenheterna (exempelvis de
skolelever som svarat på en enkät) radvis. En
enskild cell redovisar alltså en specifik
observationsenhets värde på en specifik
variabel.

\hypertarget{variabelfuxf6nstret-variable-view}{%
\section{Variabelfönstret (Variable view)}\label{variabelfuxf6nstret-variable-view}}

I detta fönster visas de olika variablerna och
deras egenskaper. Variablerna redovisas
radvis och egenskaperna kolumnvis.
Byte av fönster görs enkelt genom att klicka
nere i vänstra hörnet, där man kan välja
mellan Data View och Variable View.

\hypertarget{mer-om-variabelfuxf6nstret-variable-view}{%
\subsection{Mer om variabelfönstret (Variable view)}\label{mer-om-variabelfuxf6nstret-variable-view}}

När du öppnar ett befintligt datamaterial får du upp detta i datamatrisfönstret (Data View), antingen
med variabelvärden i text (bilden till vänster) eller som siffror (till höger). Hur man väljer att se värdena kan man ange via menyns ''View''):

I exemplet har datafilen Skol05.sav öppnats och varje rad motsvarar här en skolelev och dennes svar på
de olika frågorna i enkäten. Vid kodningen har varje svar kodats in med en siffra och dessa värden har
getts etiketter/labels.

Om du byter till Variable View (detta görs enkelt längst ner till vänster) ser du hur materialet kodats in
med etiketter (labels) som anger både en precisering av de olika variablerna och de olika variabelvärdena:

I variabelfönstret redovisas alltså variablerna radvis och dess egenskaper kolumnvis. I detta fönster
finns ett antal kolumner som är viktiga att känna till. Under den första kolumnen ''Name'' listas
variabelnamnet på varje variabel i ditt dataset. Variabelnamnet hålls vanligtvis kort och det blir därför
ofta svårt att utläsa vad variabeln mäter. Information om vad variablerna mäter och hur de är
konstruerade finns ibland i en separat kodbok eller variabellista men du kan även förtydliga dina
variabler i själva datafilen. Under kolumnen ''Label'' anges variabeletiketten, som är en kort beskrivning
av variabeln. Under kolumnen ''Values'' anges vad variabelns olika värden svarar mot.

Variabeln ''kon'' har till exempel värde 0 eller 1, där värde 0 betyder ''flicka'' och värde 1 betyder
''pojke''. För att undersöka detta markerar du cellen för den aktuella variabeln under kolumnen
''Values'' och klickar därefter på den lilla rutan till höger. Nu kommer fönstret ''Value Labels'' upp.
Om variabeln saknar etiketterade variabelvärden kan du ange dessa genom att föra in värdet i rutan
''Value'' och vad värdet står för i rutan ''Label''. Klicka därefter ''Add''. När du etiketterat samtliga
variabelvärden klickar du ''OK''.

Tänk på att variabelns värden endast behöver specificeras för variabler som befinner sig på nominaleller
ordinal skalnivå. Om du till exempel har en kontinuerlig variabel som mäter ålder i antal år
(kvotskala) är variabelvärdena i sig informativa och du behöver inte förtydliga vad dessa står för.

\hypertarget{output-fuxf6nstret-output}{%
\section{Output-fönstret (Output)}\label{output-fuxf6nstret-output}}

I detta fönster visas resultat. Det kan röra sig
om frekvenstabeller, korstabeller eller olika
typer av diagram. Till vänster finns en
översiktsmeny där du enkelt kan bläddra
mellan dina resultat.

\hypertarget{syntax-syntax-editor}{%
\section{Syntax (Syntax editor)}\label{syntax-syntax-editor}}

Vissa bearbetningar av data där man vill kombinera olika variabler kan vara enklare att göra i syntax. I syntax kan du skriva kommandon, dvs. det du vill att SPSS ska göra åt dig -- en frekvenstabell,
korstabell, omkodning av variabler mm. Alla sådana moment har sina egna kommandon.

Du kan öppna syntax via File \textgreater{} New \textgreater{} Syntax:

Kommandon kan antingen skrivas direkt i syntax eller så kan du gå via menysystemets funktioner och
klick på Paste

Klickar du på denna knapp så kommer syntax kommandot för denna omkodning att öppnas i
syntaxfönstret:

``Syntaxspråket'' tar ett tag att lära sig, och är heller inte nödvändigt för er att kunna, men att använda
sig av syntax (direkt eller via ''paste'') kan många gånger underlätta arbetet i SPSS och om du sparar
syntax får du en loggbok över det du har gjort, på så sättt är det enkelt att gå tillbaka för att se hur du
gått tillväga vid analyser, omkodningar etc. (Om du sparar de kommandon och körningar du har gjort
och har användning för i syntax, skriv gärna rubriker/kommentarer som anger vad och varför du gjort
olika moment. Obs: För att inte programmet ska missta rubriker/kommentarer för kommandon måste
du skriva en asterisk (*) före dessa och avsluta med en punkt.). Om ni är intresserade av att lära er mer om syntaxspråket rekommenderas följande sida.

\hypertarget{part-fuxf6rbereda-datamaterialet}{%
\part*{Förbereda datamaterialet}\label{part-fuxf6rbereda-datamaterialet}}
\addcontentsline{toc}{part}{Förbereda datamaterialet}

\hypertarget{databearbetning}{%
\chapter{Databearbetning}\label{databearbetning}}

En stor del av arbetet består av att bearbeta sitt datamaterial inför analys. Det finns ett antal funktioner
i SPSS med vilka man enkelt bearbetar materialet efter kodning. Kanske är du bara intresserad av att
studera vissa observationsenheter, t.ex. bara kvinnor? Eller så kanske du vill koda om variabler (slå ihop
svarskategorier eller klassindela) eller skapa nya variabler som bygger på information i två eller fler
befintliga variabler (t.ex. skapa ett index). Några funktioner för detta presenteras nedan.

\hypertarget{select-cases-analysera-enbart-vissa-observationssenheter}{%
\section{Select Cases: Analysera enbart vissa observationssenheter}\label{select-cases-analysera-enbart-vissa-observationssenheter}}

Funktionen används när du endast vill undersöka vissa observationsenheter, exempelvis endast de
flickor som ingår i skolundersökningen. Hur variabeln kön har kodats framgår av variabelförteckningen
(du hittar denna som pdf fil i Athena), alternativt kan du markera variabeln i SPSS
genom att välja Utilities \textgreater{} Variables, leta reda på Kön i listan och titta i rutan ''Variable
information''.

Följande kan då utläsas: Du finner att flickor har värdet 0 på variabeln ''kon''. För att enbart välja ut de observationsenheter som
är flickor (alltså har värdet ''0'' på variabeln ''kon'') gör du följande:

Data \textgreater{} Select Cases \textgreater{} If condition is satisfied \textgreater{} If

Lyft in den aktuella variabeln i fönstret och ange villkor (variabelvärde för att tas med), därefter:

Continue \textgreater{} OK

Tänk på att du nu har angett att kommande analyser endast ska göras för de med värde ''0'' på
variabeln ''kon''. Om du vill återgå till att analysera samtliga observationsenheter (både pojkar och
flickor), klickar du på alternativet ''All cases''. Alltså:

Data \textgreater{} Select Cases \textgreater{} All cases

Notera att det är variabeletiketten (''Kön'') och inte variabelnamnet (''kon'') som står i rullistan till
vänster i bilden ovan. Standardinställningen är att visa variabeletiketten om en sådan finns, men
ibland är det smidigare att istället visa variabelnamnet. För att göra detta högerklickar du på
variabellistan till vänster och markerar ''Display Variable Names''.

\hypertarget{split-files-separata-analyser-fuxf6r-fuxf6rutbestuxe4mda-grupper}{%
\section{Split Files: Separata analyser för förutbestämda grupper}\label{split-files-separata-analyser-fuxf6r-fuxf6rutbestuxe4mda-grupper}}

Till skillnad från ovanstående exempel där vi exkluderade pojkar för att enbart studera flickor ska vi nu studera både flickor och pojkar men separat. På så sätt kan vi exempelvis se hur utsattheten för grovt våld ser ut bland flickor respektive pojkar. Gör så här:

Data \textgreater{} Split File

Bocka för Compare groups. Välj därefer den variabel som innehåller de grupper som du vill separera dina observationer på. Det kan exempelvis vara könsvariabeln exempelvis eller födelselandsvariabeln. Tryck på OK.

De analyser du nu gör kommer att göras utifrån gruppnivå. Tänk på att om du vill åter göra aggregerade analyser (och inte på gruppnivå) så behöver du återställa gruppinställningen. För att göra det klickar du åter in dig till Split file-rutan. Tryck därefer på Reset och sedan OK. Nu är gruppindelningen borttagen och vidare analyser görs på hela datamaterialet.

\hypertarget{recode-koda-om-befintliga-variabler}{%
\section{Recode: Koda om befintliga variabler}\label{recode-koda-om-befintliga-variabler}}

Funktionen Recode kan användas till att koda om variabler (koda om befintlig variabel: ''Recode into
Same Variables'', eller till en ny variabel: ''Recode into Different Variables''). ''Recode'' används ofta då
variabeln har många kategorier/variabelvärden som forskaren vill sammanfatta till färre
kategorier/variabelvärden. Det används om man vill kategorisera en kontinuerlig variabel (t.ex.
klassindela) eller om man vill slå ihop kategorier i en kategorisk vaiabel. Variabeln offgrov i
skolundersökning (Skol05.sav) har tre värden: 0, 1 respektive 2, se frekvenstabellen nedan:

Anta att du vill skapa en variabel som anbart skiljer på utsatt respektive ej utsatt för grövre våld. Du vill
alltså skapa en ny variabel där de som svarat ja hamnar i samma kategori oavsett om varit utsatta en
gång (variabelvärde 1) eller två gånger eller fler (variabelvärde 2). Du vill med andra ord att den nya
variabeln enbart ska ha två kategorier; Nej och Ja:

Transform \textgreater{} Recode into Different Variables

För över den variabel som du vill koda om till den stora rutan med rubriken ''Numeric Variable-\textgreater Output
Variable'' och ange vad den nya variabeln ska ha för variabelnamn (''Name'') och variabeletikett
(''Label''). Klicka därefter på ''Change''. I detta fall har den nya variabeln fått namnet offgrov2.

Gå sedan vidare och öppna ''Old and new values''. Ange variabelvärdet i den ursprungliga variabeln
(''Old Value'') och vilket värde detta ska bli i den nya variabeln (''New Value''). Koda om ett värde i
taget och klicka på ''Add'' efter att du har angett det gamla och det nya variabelvärdet. I rutan ''Old \textgreater{} New:''
får du en översikt på hur du har valt att göra omkodningen. Bilden nedan visar att både
värde 1 och värde 2 i den gamla variabeln ska ha värde 1 i den nya variabeln. När du har
kontrollerat att omkodningen ser riktig ut väljer du ''Continue'' och sedan ''OK''. Kontrollera att du
har fått den nya variabeln tillagd längst ner i variabellistan (Variable view).

I exemplet ovan är det två variabelvärden i den gamla variabeln som slagits samman. Ibland har dock de
variabler du vill klassindela betydligt fler variabelvärden, och det är då osmidigt att ange varje enskilt
variabelvärde under ''Value:''. I dessa fall markerar du istället ''Range:'' och anger i den övre rutan den
lägsta klassgränsen och i den nedre rutan den högsta klassgränsen. Tänk dig till exempel att du ska
klassindela variabeln ålder i Nationella trygghetsundersökningen. Istället för att under ''Value:'' ange
vad varje enskild ålder ska tillhöra för klass kan du använda ''Range''. Ett tips: För att förtydliga vad dina
variabelvärden i den nya variabeln betyder letar du upp denna i variabelfönstret och anger detta under
''Values''.

\hypertarget{compute-att-skapa-nya-variabler-utifruxe5n-befintliga-variabler}{%
\section{Compute: att skapa nya variabler utifrån befintliga variabler}\label{compute-att-skapa-nya-variabler-utifruxe5n-befintliga-variabler}}

Med hjälp av funktionen Compute kan du skapa nya variabler. Antag att du vill slå samman -- summera -
två variabler till en variabel. Att summera variabler kan användas för att skapa summaindex. Index
innebär alltså att man summerar värdena på flera variabler till en totalsumma. Istället för flera variabler
som mäter samma underliggande fenomen (i exemplet: brott) får vi en sammanfattande variabel.
Utifrån skolundersökningen kan index exempelvis skapas utifrån frågor om brottslighet, betyg, attityder
osv (se vidare Djurfeldt m.fl. 2010/2018, Appendix 3).

I detta exempel består datamaterialet av 6 individer och två variabler -- antal stöldbrott (''Stöldbrott'')
och antal våldsbrott (''Våldsbrott'').

Anta att du vill skapa en variabel som anger det totala antalet brott:

Transform \textgreater{} Compute Variable

Namnge den nya variabeln i Target variable. I detta fall döps den nya variabeln till ''Totbrott''. I rutan
''Numeric expression'' anges funktionen för att skapa den nya variebeln, i detta fall anges vilka variabler
som skall summeras (+):

Klicka därefter ''Continue'' och ''Ok''. Den nya variabeln placeras längst till höger i datamatrisen (''Data
view'') och längst ner i variabelfönstret (''Variable view'').
Summering kan också göras genom att ange funktionen ''sum(Våldsbrott, Stöldbrott)'', skillnaden
mellan de två sätten är att missing values (internt bortfall) behandlas olika: Med funktionen ''sum'' får
inte de observationsenheter som har missing på någon av de variabler som summeras missing på den
nya variabel du skapar.

Du kan även välja att skapa en variabel som anger medelvärdet på de ingående variablerna i indexet.
Detta gör du enklast med funktionen ''mean()''. En fördel med detta val är att du även kan ange hur
många av de ingående variablerna som måste ha valida värden för att individen ska få ett värde på
indexet. Anta att du vill skapa ett index som mäter vänners antisociala attityder och du vill inkludera
följande fem variabler (Skol05): Har någon av dina kompisar 1) tagit något utan att betala i en affär?
(''snattkom''), 2) förstört någonting? (''kompvand''), 3) brutit sig in någonstans? (''kompinbro''), 4) slagit
ner någon? (''kommissh''), 5) åkt fast för polisen? (''komaktfa''). Den svarande kan välja att svara ja eller
nej där 1 är ja och 0 är nej. Ett medelvärdesindex kommer därför att variera mellan 0 och 1 där 1
betyder att individen har svarat ja på de ingående enkätfrågorna och 0 betyder att individen har svarat
nej på de ingående enkätfrågorna. Följande funktion skulle först summera alla individens värden på de
fem ingående variablerna och därefter dividera summan med fem, d.v.s. antalet variabler:

\begin{Shaded}
\begin{Highlighting}[]
\NormalTok{(snattkom }\SpecialCharTok{+}\NormalTok{ kompvand }\SpecialCharTok{+}\NormalTok{ kompinbro }\SpecialCharTok{+}\NormalTok{ kommissh }\SpecialCharTok{+}\NormalTok{ komaktfa) }\SpecialCharTok{/} \DecValTok{5}
\end{Highlighting}
\end{Shaded}

Men ovanstående funktion kräver att individen har valida värden på samtliga fem ingående variabler,
om en enda variabel saknar ett valitt värde så kommer SPSS inte att beräkna ett indexvärde för den
individen. Detta gör inte så mycket om de ingående variablerna har ett litet internt bortfall för ungefär
samma individer. Problemet uppstår när det finns ingående variabler med stort internt bortfall och/eller
om det interna bortfallet är fördelat på en stor andel av individerna. Detta resulterar i att ditt index får
ett mycket stort internt bortfall vilket kan skapa osäkerhet och precisionsförlust i kommande analyser.
En smart lösning på problemet är att bestämma ett visst antal variabler som individen måste ha svarat
på för att ingå i indexet och beräkna indexvärdet endast på dessa variabler. Säg att du väljer att endast
tre av de fem ingående enkätfrågorna måste ha besvarats för att ett indexvärde ska beräknas. Detta gör
du genom att placera en punkt och siffran 3 efter själva mean-funktionen enligt koden nedan:

\begin{Shaded}
\begin{Highlighting}[]
\FunctionTok{MEAN.3}\NormalTok{(snattkom, kompvand, kompinbro, kommissh, komaktfa)}
\end{Highlighting}
\end{Shaded}

Om du anger mean-funktionen utan att specificera hur många variabler som ska ha valida värden så är
utgångspunkten att åtminstone två variabler har valida värden. Men detta kan vara lite väl skakigt
(försämra begreppsvaliditeten, se Bryman) på ett index som består av många variabler. Tänk därför på
hur många variabler som det är rimligt att en individ ska ha valida värden på för att ingå i indexet. Detta
är ett val som forskaren ska göra och inte SPSS.

Bra att känna till är att oavsett om du väljer en summa- eller medelvärdesfunktion så kommer
fördelningen i indexet (när du t.ex. vill beskriva variabeln i en frekvenstabell) att vara identisk förutsatt att beräkningen tar hänsyn till lika många variabler. Skillnaden kommer endast att vara skalan på vilket indexet mäts.

\hypertarget{variabelkonstruktion-med-hjuxe4lp-av-syntax}{%
\chapter{Variabelkonstruktion med hjälp av syntax}\label{variabelkonstruktion-med-hjuxe4lp-av-syntax}}

Ett exempel på när bearbetningar är bra att göra i SPSS är när man vill kombinera olika variabler och
svaralternativ för att skapa en ny variabel. En funktion för detta är Compute (se kapitlet Databearbetning), en annan är funktionen If. Med If kan man precisera villkor för variabelvärden utifrån en eller flera variabler.

Se kapitlet Viktiga rutor i SPSS för en kort beskrivning av syntaxen.

I skolunderökningen (Skol05.sav) finns tre variabler som anger födelseland: För eleven själv
(''fodland1''), för dennes mor (''mfodland'') respektive för dennes far (''pfodland''). Utifrån dessa tre
variabler kan man skapa en varaibel som anger utländsk bakgrund -- ''utl\_bkgr'' - som får värdet 0 om
eleven själv och dennes föräldrar är födda i Sverige och värdet 1 om eleven själv och någon av dennes
föräldrar är födda i annat land.

Öppna syntax: File / New / Syntax

Skriv ett kommande som anger hur variabeln ska skapas i stil med nedanstående kod. Granska gärna kommandot nedan och fundera över dess innebörd. Täcker villkoren in alla? Finns andra möjliga definitioner?

\begin{Shaded}
\begin{Highlighting}[]
\ControlFlowTok{if}\NormalTok{ (}\AttributeTok{fodland1=}\DecValTok{0}\NormalTok{ and }\AttributeTok{mfodland=}\DecValTok{0}\NormalTok{ and }\AttributeTok{pfodland=}\DecValTok{0}\NormalTok{) utl\_bkgr}\OtherTok{=}\FloatTok{0.}
\ControlFlowTok{if}\NormalTok{ (}\AttributeTok{fodland1=}\DecValTok{1}\NormalTok{ or }\AttributeTok{mfodland=}\DecValTok{1}\NormalTok{ or }\AttributeTok{pfodland=}\DecValTok{1}\NormalTok{) utl\_bkgr}\OtherTok{=}\FloatTok{1.}
\NormalTok{execute.}
\end{Highlighting}
\end{Shaded}

Markera programsatsen och klicka på den gröna pilen. I datafilen finns nu den nya variabeln ''utl\_bkgr''. Nästa steg är att kontrollera så att den konstruerade
variabeln blivit korrekt: Ta fram en frekvens (se vidare nedan), ser fördelninge rimlig ut om du jämför
med ursprungsvariablerna? Kontrollera också gärna ett par av dina observationsenheter -- utifrån de
värden de har på de tre ursprungliga variablerna, har det blivit korrekt? Därefter går du vidare med att
sätta labels på variabeln. Observera att när du konstruerar nya variabler är det självfallet viktigt att dessa är genomtänkta och välmotiverade.

\hypertarget{part-analysera-data-i}{%
\part*{Analysera data I}\label{part-analysera-data-i}}
\addcontentsline{toc}{part}{Analysera data I}

\hypertarget{beskrivande-statistik}{%
\chapter{Beskrivande statistik}\label{beskrivande-statistik}}

Innan ens hypoteser sätts på prov utförs en så kallad deskriptiv analys (även kallad explorativ dataanalys) där ens datamaterial och i synnerhet de aktuella variabler sammanfattas på olika kvantitativa sätt. Oftast görs det med hjälp av att frekvenstabeller, central- och spridningsmått.

\hypertarget{frekvenstabeller-central--och-spridningsmuxe5tt}{%
\section{Frekvenstabeller, central- och spridningsmått}\label{frekvenstabeller-central--och-spridningsmuxe5tt}}

Det är vanligt att man inleder en studie med att studera hur observationsenheterna fördelar sig med
avseende på en enskild variabel. Viktiga verktyg i detta ändamål är frekvenstabeller, centralmått och
spridningsmått. Anta att du studerar den Nationella trygghetsundersökningen (NTU 2013-15 M2.sav)
och vill ha information om oro över brottsligheten i samhället:

Analyze \textgreater{} Descriptive statistics \textgreater{} Frequencies

Börja med att söka upp den variabel du är intresserad av i rullistan till vänster (kom ihåg att du kan
välja att visa variabelnamn eller variabeletiketter genom att högerklicka på listan). Därefter markerar
du variabeln och flyttar över den till den högra rutan genom att använda pilen mellan rutorna

alternativt dubbelklicka på variabeln. Om du nu väljer alternativet ''OK'' kommer SPSS att producera en
frekvenstabell på variabeln. Detta är standardvalet (som du kan se är valet ''Display frequency tables''
markerat), men ofta vill man sammanfatta sin variabel lite mer utförligt. Längst till höger finns
möjligheter att ytterligare specificera vad du vill få fram för statistik. Till exempel kan du genom att
klicka på ''Statistics'' välja central- och spridningsmått

Skalnivån (mätnivån) för frågan om oro är på ordinal nivå varför vi i detta fall nöjer oss med en
frekvenstabell:

Av frekvenstabellen kan vi utläsa mer specifikt hur många individer och hur stor andel som uppger sig
vara oroliga för brottsligheten i samhället. Vi kan till exempel se att endast 24 procent svarar att de
inte alls är oroliga.

\hypertarget{kort-om-grafiska-tekniker}{%
\section{Kort om grafiska tekniker}\label{kort-om-grafiska-tekniker}}

I syfte att beskriva våra resultat i grafisk form kan man även välja ''Graphs'' i huvudmenyn. Genom
alternativet ''Chart Builder'' kan du välja på en mängd olika diagramtyper som på bästa sätt beskriver
din/a variabel/ler. Alltså:

Graphs \textgreater{} Chart Builder

Vanliga diagramtyper för att beskriva enskilda variabler är stapeldiagram (''Bar chart''), cirkeldiagram
(''Pie chart'') och histogram. Med detta verktyg kan du på grafisk väg även studera eventuella samband
mellan två variabler. Vi återkommer senare till att åskådliggöra samvariationen mellan två kontinuerliga
variabler genom att ta fram ett så kallat spridningsdiagram (''Scatter plot'').

Det går även att i samband med skapandet av en frekvenstabeller ange att man vill ha ett diagram. De diagram som då finns att tillgå är stapeldiagram, cirkeldiagram och histogram. Klicka er fram till rutan för frekvenstabeller:

Analyze \textgreater{} Descriptive statistics \textgreater{} Frequencies

Mata sedan in den variabel som ni vill skapa ett diagram över. Klicka därfter på knappen Charts. Välj sedan vilket diagram ni önskar att få fram. Klicka därefter på Continue och sedan OK. I output:en ska ni nu ha fått fram en frekvenstabell samt den figur ni har valt.

\hypertarget{part-analysera-data-ii}{%
\part*{Analysera data II}\label{part-analysera-data-ii}}
\addcontentsline{toc}{part}{Analysera data II}

\hypertarget{bivariat-analys-att-studera-samvariationen-mellan-tvuxe5-variabler}{%
\chapter{Bivariat analys: Att studera samvariationen mellan två variabler}\label{bivariat-analys-att-studera-samvariationen-mellan-tvuxe5-variabler}}

Under denna kurs kommer du vilja undersöka huruvida det finns ett samband mellan två variabler. Hur sambandet undersöks bestäms helt uifrån vilken datanivå era variabler har.

\hypertarget{samband-mellan-tvuxe5-kategoriska-variabler}{%
\section{Samband mellan två kategoriska variabler}\label{samband-mellan-tvuxe5-kategoriska-variabler}}

Under förutsättning att variablernas skalnivåer är nominal- eller ordinalskala (ej intervall- eller
kvotskala) analyseras sambandet vanligtvis genom att studera de båda variablerna i en korstabell (''Crosstab''). Gör följande:

Analyze \textgreater{} Descriptive statistics \textgreater{} Crosstabs

I detta fönster har du likt tidigare en rullista till vänster som innehåller samtliga variabler i
datamaterialet. Innan du fortsätter är det viktigt att du, med hänvisning till din frågeställning, har gjort klart vilken variabel som är tänkt att påverka den andra. Beroende variabel placeras i radled (''Row(s)'') och oberoende variabel placeras i kolumnled (''Column(s)''). Vi kan t.ex. vara intresserade av huruvida oro för brottsligheten i samhället skiljer sig åt efter kön. Vi gör då en korstabell med variablerna S4 (Oro brottslighet) Kön. Identifiera variablerna i rullistan till vänster och för sedan över dessa till ''Row(s)'' respektive ''Column(s)'' genom att använda pilarna. I detta fall gjordes alltså antagandet att Kön är
oberoende. Som framgår finns ytterligare funktioner/alternativ. För kursen relevanta rutor är här
''Statistics'', ''Cells'' och ''Format''. Under ''Statistics'' kan man välja mellan ett flertal olika
sambandsmått och signifikanstest. Vi återkommer till sambandsmått och signifikanstest, nu ligger
fokus på att konstruera en korstabell som kan möjliggöra tolkningen av om och i så fall hur våra
variabler är relaterade till varandra. För att underlätta denna tolkning väljer du först alternativet
''Cells''.

Att sammanställa tabellen endast med antal observationer i varje cell gör en jämförelse svår. Under
rubriken ''Percentages'' är det är möjligt att markera om korstabellen ska sammanställas med rad-
(''Row), kolumn- (''Column''), och/eller totalprocent (''Total''). Radprocent innebär att summera
varje rad till 100 procent, medan totalprocent innebär att redovisa hur stor andel varje cell utgör av
samtliga observationer. Här har vi valt att markera kolumnprocent, med vilket avses att kolumnerna
summeras upp till 100 procent.

När den oberoende variabeln är placerad i kolumnled och den beroende variabeln i radled möjliggör
valet av kolumnprocent tolkningen av huruvida det verkar finnas ett samband mellan variablerna --
om oro för brottsligheten i samhället skiljer sig åt mellan män och kvinnor. Eftersom vi konsekvent
väljer att placera våra variabler på detta sätt kommer vi alltså i syfte att utreda ett eventuellt samband
alltid vilja redovisa korstabellen med kolumnprocent. Klicka på ''Continue'' när du gjort ditt val.
Slutligen ska vi ta en titt på alternativet ''Format''. Här kan du välja om radledet ska redovisas i stigande
(''Ascending'') eller fallande (''Descending'') ordning. I syfte att tolka riktningen på sambandet kan valet
av stigande eller fallande ordning underlätta, men detta är en smaksak och du kommer oavsett val
kunna göra samma tolkning av korstabellen i syfte att utreda ett eventuellt samband. I detta fall har
standaralternativet stigande ordning valts, vilket innebär att korstabellen i radled kommer att
sammanställas med det lägsta värdet överst. Klicka på ''Continue'' när du gjort ditt val.

Klicka därefter ''OK'' så producerar SPSS en korstabell över relationen mellan kön och oro för brottsligheten.
Resultatet visas i output fönstret. Så som vi har valt att sammanställa vår korstabell (oberoende variabel
i kolumnled och beroende variabel i radled och sammanställd med kolumnprocent) är det nu möjligt att
se om om det verkar finnas ett samband mellan kön och oro för brottsligheten i samhället. Nästa steg är
att tolka korstabellen. Tekniken är att jämföra kategorierna i den oberoende variabeln radvis.

I korstabellen kan vi se att 31 procent av männen jämfört med 17,5 procent av kvinnorna svarat att de
inte alls är oroliga. Det verkar alltså som att det finns ett samband i den meningen att kvinnor är mer
oroliga för brottsligheten än män.

Eftersom vi har en variabel på ordinal nivå (oro) och en på nominal nivå (kön), kan vi inte uttala oss om
riktningen på sambandet, dvs. om det rör sig om ett positivt eller negativt samband (hur vi kodat
variabeln kön, dvs. vilket kön som kodats som 1 eller 2, är ju godtyckligt).

\hypertarget{samband-mellan-tvuxe5-numeriska-variabler}{%
\section{Samband mellan två numeriska variabler}\label{samband-mellan-tvuxe5-numeriska-variabler}}

När vi har att göra med variabler som befinner sig på intervall- eller kvotskala är varken korstabell eller
ovan nämnda sambandsmått lämpliga verktyg för att utreda ett eventuellt samband. Föreställ dig till
exempel att undersöka sambandet mellan ålder och brott i en korstabell, där respondenten i
enkätundersökningen har fått ange sin ålder och även självskatta antalet begångna brott under det
senaste året -- det skulle resultera i en enorm korstabell eftersom varje specifik ålder- och
brottkombination kräver sin egen cell.

I detta fall är istället ett spridningsdiagram (''Scatter plot'') lämpligt att använda för att studera huruvida
ett samband verkar föreligga. Gör följande:

Graphs \textgreater{} Chart builder

Under ''Gallery'', klicka på ''Choose from'', välj ''Scatter / Dot'' och dra ''Simple Scatter'' upp till rutan
''Chart Preview''. Dra din oberoende variabel till rutan för x-axeln och din beroende variabel till rutan för
y-axeln. Klicka därefter ''OK''.

Vi kan även använda sambandsmått för att beräkna styrka och riktning på sambandet. När vi har att
göra med två kontinuerliga variabler är sambandsmåttet Pearson's r (korrelationskoefficienten r)
lämpligt att använda. Gör följande:

Analyze \textgreater{} Correlate \textgreater{} Bivariate

För över de variabler du vill korrelera till rutan ''Variables'' och markera Pearson's r. Klicka därefter
''OK''. Precis som tidigare nämnda sambandsmått varierar Pearson's r på en skala mellan -1 och +1, där
0 indikerar att det inte finns ett samband medan -1 anger ett perfekt negativt samband och +1 anger ett
perfekt positivt samband.

Ett alternativ till att studera relationen mellan ålder och brott på ovanstående vis är att klassindela de
båda kontinuerliga variablerna till kategoriska variabler med hjälp av recode-kommandot (se under
Databearbetning). Man kan tänka sig att klassindela ålder till de tre klasserna ''ungdom'', ''ung vuxen''
samt ''vuxen'', samt brott till de tre klasserna ''inga brott'', ''1-2 brott'', ''3 eller fler brott''. På det sättet
skulle vi konstruera två variabler på ordinal skalnivå utav de två ursprungliga variablerna på kvotskala.
Därmed kan vi med de nya variablerna studera relationen mellan ålder och brott i en korstabell (i detta
fall med nio celler) och med de sambandsmått som är lämpliga för variabler på ordinal skalnivå. Tänk på
att variabler på högre skalnivå alltid kan transformeras till variabler på lägre skalnivå.

\hypertarget{sambandsmuxe5tt}{%
\section{Sambandsmått}\label{sambandsmuxe5tt}}

I ovanstående exempel kunde vi, genom att tolka korstabellen, se att ett samband verkar föreligga
mellan kön och oro för brottslighet. Ibland vill man även uttala sig om sambandets styrka och i detta
syfte är användningen av sambandsmått bra. I de fall som sambandets riktning är tolkningsbart ger
sambandsmåttet även denna information. Statistiker har tagit fram olika sambandsmått som gäller för
variabler som befinner sig på olika skalnivåer. För att välja sambandsmått börja med följande:

Analyze \textgreater{} Descriptive statistics \textgreater{} Crosstabs

Placera din oberoende variabel i kolumnled och din beroende variabel i radled. Välj även, precis som
tidigare, att sammanställa korstabellen med kolumnprocent under alternativet ''Cells''. Klicka därefter
på ''Statistics''. Här får vi en viss vägledning av SPSS när det gäller vilka sambandsmått som är lämpliga att använda för våra variabler beroende på datanivå.

\textbf{Överkurs:}
Vi kommer i denna kurs inte gå på djupet med de sambandsmått som finns och hur man ska tolka resultaten från de. Vill ni dock ha fördjup. Om ni dock önskar att läsa er in på vilka sambandsmått som finns, när ni ska använda de och vilka sambandsmått som passar till vilken typ av variabler rekommenderar vi följande artikel.

\hypertarget{part-analysera-data-iii-hypotespruxf6vning}{%
\part*{Analysera data III: Hypotesprövning}\label{part-analysera-data-iii-hypotespruxf6vning}}
\addcontentsline{toc}{part}{Analysera data III: Hypotesprövning}

\hypertarget{att-vuxe4lja-ruxe4tt-test-vid-hypotespruxf6vning}{%
\chapter{Att välja rätt test vid hypotesprövning}\label{att-vuxe4lja-ruxe4tt-test-vid-hypotespruxf6vning}}

Samhällsvetare arbetar nästan alltid med urval dragna ur en population men vill generalisera till hela
populationen. Den gren inom statistik som hjälper oss att göra detta kallas statistisk inferens.
Arbetsgången att pröva de samband vi är intresserade av är att ställa upp nollhypotes och mothypotes,
där nollhypotesen uttrycker att det inte finns en skillnad medan mothypotesen uttrycker att det finns
en skillnad. Det vi prövar är något förenklat om skillnaderna och de samband vi undersöker är tillräckligt
stora för att kunna antas gälla i populationen. Om så är fallet förkastar vi nollhypotesen.

Under detta avsnitt följer två vanliga metoder för hypotesprövning för att avgöra om ett samband är
signifikant: chi2 och t-test. Chi2 används när oberoende och beroende variabel befinner sig på nominal
eller ordinal skalnivå, medan t-test används när den oberoende variabeln befinner sig på nominal eller
ordinal skalnivå och har två kategorier medan den beroende variabeln är kontinuerlig och befinner sig
på intervall- eller kvotskala.

Oavsett vilket av dessa signifikanstest som är lämpligt kommer ni på liknande sätt att tolka det p-värde
som SPSS beräknar för att avgöra om ett samband är signifikant eller inte. Är en uppmätt skillnad
''verklig'' eller kan den bero på slumpen? Ett p-värde under 0.05 är signifikant på fem procents nivå, ett
värde under 0.01 på en procents nivå och ett värde under 0.001 på en promilles nivå. Vanligt är att
ställa upp fem procent signifikansnivå som gräns för vad som ska anses vara ett signifikant resultat. Det
innebär att endast i fem fall av hundra skulle urvalet visa på en skillnad som egentligen inte existerar i
populationen. Ett annat sätt att uttrycka detta på är att risken att vi förkastar en sann nollhypotes är
fem procent (kallas även typ 1-fel).

Vilket statistiskt test man behöver använda för att utföra en hypotesprövning bestäms helt utifrån de variabler man använder sig av.
Listan på vilka statistiska test som finns att tillgå är lång men det finns några få som förekommer ofta eller som är praktiska att känna till
eftersom de underlättar förståelsen för andra statistiska test. I kursen Metod II kommer ni att stifta bekanskap med tre statistiska test:
t-test, chi2 och regression. Den sistnämnda kan i sin tur delas in i två kategorier: enkel och multipel.

Med hjälp av nedanstående flödesschema kan ni avgöra vilket statistiskt test som lämpar sig bäst för de variabler som ni önskar att analysera. Observera att ett ytterligare
test finns omnämnt i flödeschemat (linjär sannolikhetsmodell/logistisk regresssion) men som vi ej kommer gå igenom under kursen gång.
Den finns med eftersom den introduceras i samband med den kvantitativa metodkursen på avancerad nivå.

\hypertarget{chi2-test}{%
\chapter{Chi2-test}\label{chi2-test}}

I exemplet ovan kunde vi se att oro för brottsligheten i samhället skiljer sig för män och kvinnor. Utifrån
vår korstabell kan vi dock endast uttala oss om förhållandet i urvalet. Vad vi nu vill ta reda på är om detta
samband mellan kön och oro för brottsligheten är tillräckligt tydligt för att kunna antas gälla i
populationen, dvs. i befolkningen. Eftersom vi har två kategoriska variabler är chi2 ett lämpligt
signifikansmått. Proceduren i beräkningen av chi2 är att jämföra observerade frekvenser med de
frekvenser som skulle förväntas om det inte fanns någon skillnad. Genom denna jämförelse får vi fram
ett chi2-värde som vi i nästa steg kan jämföra med en så kallad chi2-fördelning för att ta reda på
huruvida detta värde överstiger ett kritiskt värde som motsvarar en given signifikansnivå. Genom denna
procedur skulle vi utifrån vår korstabell kunna räkna ut huruvida sambandet är signifikant eller inte,
men som tur är har vi SPSS till vår hjälp. Arbetsgången är följande:

Analyze \textgreater{} Descriptive statistics \textgreater{} Crosstabs

Välj precis som tidigare att placera ''kön'' i kolumnled och ''S4'' i radled. Välj även, precis som tidigare,
att sammanställa korstabellen med kolumnprocent under alternativet ''Cells''. Klicka därefter på
''Statistics'' och markera alternativet ''Chi-square''. Klicka på ''Continue'' och därefter ''OK''. Utöver korstabellen får du nu även en tabell som ger
information om ditt begärda Chi2-test.

Titta på raden ''Pearson Chi-Square'' och kolumnen ''Asymp. Sig. (2-sided)''. Här kan du se att ditt pvärde
är mindre än 0.01. Skillnaderna är alltså så pass stora att dessa på en procents signifikansnivå kan
antas gälla i populationen. Observera att du även har ett stort antal observationer, vilket även det har
betydelse för dina möjligheter att finna skillnader som går att generalisera (gör att du får ett högre Chi2
värde). Sambandet är signifikant på en procents signifikansnivå och vi kan alltså generalisera våra
resultat till att gälla för hela populationen.

Observera dock att SPSS anger att 0 celler har förväntade värden som understiger fem. Som regel gäller
att Chi2-testet är ogiltigt om 20\% av cellerna har ett förväntat värde mindre än 5 eller en cell har ett
förväntat värde mindre än 1 då korstabellen är större än 2x2. Om korstabellen är 2x2 får ingen av
cellerna ha ett förväntat värde som är mindre än 5. I detta fall är dock signifikanstestet giltigt. Om chi2-
testet skulle visa sig vara ogiltigt kan lösningen vara att klassindela sina variabler med hjälp av
kommandot Recode (se ovan under avsnittet Databearbetning).

\hypertarget{t-test}{%
\chapter{T-test}\label{t-test}}

När du vill pröva om en skillnad mellan två grupper är signifikant och utfallsvariabeln är kontinuerlig är
ett oberoende t-test tillämpligt. Här är proceduren att jämföra medelvärden mellan två grupper och
utifrån den uppmätta skillnaden ta reda på om den är tillräckligt stor för att antas gälla i populationen.
Anta att du är intresserad av om det finns någon skillnad i pojkars och flickors medelbetyg. Du vill
jämföra ett sammanfattande mått på deras betyg i kärnämnena svenska, engelska och matematik.
Betygsvariablerna är på ordinalskala och antar värden 0-3, se frekvenstabell nedan för betyg i
svenska:

Du skapar ett betygsindex (''Mbetyg'') som varierar mellan 0-3 genom att summera de tre betygen
och dividera med tre med hjälp av funktionen compute (se ovan under databearbetning).För att göra ett oberoende t-test, där du prövar om det finns en signifikant skillnad mellan pojkars och
flickors genomsnittliga betyg, är tillvägagångssättet följande:

Analyze \textgreater{} Compare Means \textgreater{} Independent Samples T-test

Oberoende variabel är i detta fall ''kon'' medan den beroende variabeln är ''Mbetyg''. Flytta den
beroende variabeln till rutan ''Test Variable(s)'' och den oberoende variabeln till rutan ''Grouping
Variable''. Därefter måste du definiera vilka grupper inom denna variabel som du är intresserad av.
Genom att i variabelfönstret undersöka variabeln ''kon'' ser du att variabelvärde 0 står för ''flickor''
och variabelvärde 1 står för ''pojkar''. Klicka på ''Define Groups'' och ange dessa variabelvärden i
rutorna ''Group 1:'' respektive ''Group 2:''. Klicka därefter ''Continue''.

Klicka ''OK'' och gå till Output-fönstret. Du får nu fram resultaten i två tabeller. I den första tabellen
redovisas deskriptiva mått som antalet i respektive grupp samt respektive grupps medelvärde och
standardavvikelse på den beroende variabeln.

Vi ser att medelvärdet på betygsindex är 1,59 för flickor och 1,39 för pojkar. Flickor har alltså högre
betyg. Men är denna skillnad signifikant, d.v.s. kan anta att den även existerar mellan pojkar och
flickor i populationen? För att avgöra det tittar vi på nästa tabell:

I tabellen ovan redovisas ett antal relevanta mått för det oberoende t-testet. ''Mean Differance'' är
skillnaden mellan de båda medelvärdena. I detta fall är skillnaden i genomsnittligt betyg 0,196.
Frågan är emellertid om skillnaden är tillräckligt stor för att fastställa att medelvärden skiljer sig åt i
populationen? P-värdet vid ett tvåsidigt hypotestest går att finna under rubriken ''Sig. (2-tailed)''. Ett
värde under 0,05 är signifikant på fem procents nivån. Här har vi ett värde på 0,00 vilket understiger
denna gräns. Vi kan således förkasta nollhypotesen som uttryckte att det genomsnittliga betyget är
samma för pojkar och flickor. (se vidare Djurfeldt m.fl. 2010/2018, s 230ff).

När vi gått igenom t-test på lektionerna har ett antagande gjorts om att de båda grupperna har samma
varians. Det finns emellertid två olika test - med eller utan antagande om lika varians (equal variance
assumed / equal variance not assumed). Skall man gå korrekt tillväga kontrollerar man att antagandet
håller. Det görs med Levene´s test for Equalite of Variances. F är kvoten av de båda gruppernas varians
och om denna kvot inte är lika med 1 kan det signalera att antagandet om lika varians inte håller. Om
sannolikheten för detta F-värde är mindre än 0,05 drar vi slutsatsen att skillnaden mellan urvalens
varians reflekterar en skillnad i populationernas varians. Om så är fallet, vilket det emellertid inte är här (P = 0,10) går vi till ''Equal variance not assumed''. ''Equal variance assumed'' är mer restriktivt vilket
betyder att det är svårare att få ett signifikant resultat. Därför kan ni lika gärna använda detta test.

\textbf{Överkurs:} Enklaste sättet att förstå hur det ser ut när variansen är den samma för två grupper alternativt olika för två grupper är att illustrera gruppernas varians. I figurerna här nedan har vi grupp röd och grupp blå. Grupp röd har i genomsnitt 100 i värde i både den övre och nedre figuren. Grupp blå har i genomsnitt 130 i både den övre och nedre figuren. I den övre figuren har röd en standardavvikelse på 15 och blå har en standardavvikelse på 30. I nedre figuren har både röd och blå 15 i standardavvikelse.

Som vi kan se i den övre figuren är alltså spridningen (och därför variansen) större för grupp blå än för grupp röd. I nedre figuren är spridningen den samma (och därför variansen) för båda grupperna.

Om vi ska utföra ett t-test med den data som används för den övre figuren komer SPSS att ange: ''Equal variance not assumed''.

Om vi ska utföra ett t-test med den data som används för den nedre figure kommer SPSS att ange: ''Equal variance assumed''.

\hypertarget{enkel-regression}{%
\chapter{Enkel regression}\label{enkel-regression}}

I nästa steg är vi intresserade av att se sambandet mellan två numeriska variabler. Eftersom vi enbart använder oss av två variabler och båda är numeriska är enkel regression ett lämpligt statistiska test. Med hjälp av enkel regression kan vi predicera värden eller utröna vilken effekt en variabel har påen annan. Vad det innebär att predicera och hur effekter redogörs kommer att exemplifieras här nedan.

Vi är intresserade av att se vilket samband det finns mellan utomhustemperatur (oberoende variabel) och antal anmälda brott i Stockholm (beroende variabel). Ett annat sätt att formulera forskningsfrågan är om vi kan predicera polisanmäld brott utifrån utomhustemperatur.

Vår regressionsmodel kan uttryckas i form av följande matematiska formel:

\(Y_{brott} = b_0 + b_{celcius}\)

Det vår modell säger är att vikan predicera polisanmälda brott utifrån vårt intercept/konstant (b0) och utomhustemperaturen.

Hypotesen är - utifrån rutinaktivitetsteorin - att ju varmare det är, desto fler personer vistas utomhus vilket leder till fler brottstillfällen. Datamaterialet vi använder för denna forskningsfråga är materialet från det fiktiva datasetet som innehåller uppgifter om glassförsäljning, anmälda brott, utomhustemperatur och förekomsten av regn (ice\_cream.sav).

Det första vi gör är att ta fram ett spridningsdiagram (scatter plot) med en regressionslinje. Hur man tog fram ett spridningsdiagram gick vi igenom under kapitlet som berörde bivariat analys men här kommer en repetition. Gör följande:

Graphs \textgreater{} Chart builder

Under ''Gallery'', klicka på ''Choose from'', välj ''Scatter / Dot'' och dra ''Simple Scatter'' upp till rutan ''Chart Preview''. Dra din oberoende variabel till rutan för x-axeln och din beroende variabel till rutan för y-axeln. Klicka i Total under Linear Fit Lines för att inkludera en regressionslinje till din figur. Klicka därefter ''OK''.

Som vi kan se med hjälp av regressionslinjen i spridningsdiagramet så verkar det finnas i vårt datamaterial ett starkt positivt samband mellan utomhustemperatur och polisanmäld brottslighet; ju varmare det är desto fler brott polisanmäls.

Frågan är dock om vi kan generalisera detta samband eller om det kanske kan ha uppstått av ren slump. Detta kan vi avgöra med hjälp av en regressionsanalys. Gör följande:

Analyze \textgreater{} Regression \textgreater{} Linear

I rutan som dyker upp matar ni in den oberoende variabeln - utomhustemperatur - i fältet Block 1 of 1
och den beroende variabeln - polisanmälda brott - i fältet Dependent. Tryck därefter ok.

Vi kan utifrån ovanstående resultat se flera saker värda att notera. Vi börjar med informationen i rutan ``Model Summary''. De mått som där är viktigast att titta på är Square R och Adjusted R Squared. Det anger hur stor andel varians/variation i vår beroende variabel som vår oberoende kan förklara. Det vill säga, hur stor andel variation i polisanmälda brott från dag till dag kan förklaras utifrån uthomhustemperaturen. Skillnaden mellan R Square och Adjusted R Square'' är att det sistnämnda måttet tar hänsyn till antalet oberoende variabler som ingår i ens regression. Om man inkluderar många oberoende variabler kan ''R Square'' överskatta den förklarade variansen och därför brukar man generellt använda sig av Adjusted R Square när man har fler än en oberoende variabel.

Eftersom vi enbart har en oberoende variabel så tittar vi på R Square vilket visar att 46.4\% (översätter man andelen 0.464 till procent blir det 46.4\%) av variationen i polisanmäld brott kan förklaras med hjälp av vår oberoende variabel utomhustmperatur.

I ANOVA-rutan kan vi se att vår regressionsmodell är statistisk signifikant p \textless{} 0.001 (skall ej förväxlas med den enskilda oberoende variablelns p-värde!). Det betyder att vår regressionsmodell innehållande en oberoende variabel hjälper oss att förstå den beroende variabeln bättre än enbart informationen från en modell utan en oberoende variabel (det vill säga utifrån enbart informationen om genomsnittlig antal polisanmälda brott).

Den absolut viktigaste rutan är Coefficiens-rutan vilket visar vilken effekt den oberoende variabeln har på den beroende och om denna effekt är signifikant. Vi borjar dock först med att titta på Constant vilket är vår intercept och dess koefficient vilket går att utläsa under Unstandardized B. I vårt exempel kan vi se att interceptet är -609.145. Det betyder att det predicerade värdet för vår beroende variabel är -609.145 när vår oberoende variabel antar värdet 0. Eller mer specifikt uttryckt: när utomhustemperaturen är noll grader begås det -609 brott.

\textbf{Överkurs:}
Även om interceptet är matematiskt korrekt så kan den vara orealistisk. I vårt exempel är det ett intercept som är mindre realistisk eftersom vi vet att även vid noll grader så begås det brott. Den orsak till varför vi får ett osannolikt intercept (men som är matematisk korrekt!) är att vi enbart samlat in data under sommaren och inte under vinterhalvåret. Vår modell försöker så gott det går att predicera hur det skulle se ut vid noll grader men eftersom vi saknar data för höst och vinter så ger modellen oss icke-realistiska värden. Detta är dock inget problem eftersom vi är ute efter att se sambandet mellan utomhustemperatur och brott under sommaren

Interceptets värde kan vi också se i ovanstående spridningsdiagram om vi anger att diagramets x-axel ska börja på 0 celsius. Nedanstående spridningsdiagram är identiskt med ovansående men med enda skillnaden att x-axeln börjar på 0 celcius. Notera att vi ine har förändrat något beträffande vår data utan endast förlängt diagramets x-axel. Vi kan där se att interceptet - det vill säga där regressionslinjen korsar Y-axeln - är vid -609.145.

Vidare kan vi se att koefficienten för vår oberoende variabel är 39.894 och under Sig. kan vi se att vårt p-värde är mindre än 0.001. Det betyder att för varje enhets ökning av temperaturvariabeln (alltså för varje temperaturgrads ökning) så ökar antalet polisanmälda brott med 39.894 och eftersom p-värdet är mindre än 0.05 (vilket är vårt alfavärde) så är denna effekt statistiskt signifikant.

Med hjälp av ovanstående information kan vi predicera hur många brott som kommer polisanmälas under en dag då det exempelvis är 25 grader utomhus. För att göra det tar vi vårt intercept, adderar därefter produkten av koefficenten för vår oberoende variabel och 25 (vilket är den temperatur som vi exemplefierar med).

\(Y_{brott} = b_0 + b_{celcius}\)

\(Y_{brott} = -609.145 + (39.894*25)\)

\(Y_{brott} = 388.205\)

Vi predicerar alltså att det i genomsnitt kommer polisanmälas 388 brott under en dag då utomhustemperaturen är 25 grader.

\hypertarget{multipel-regression}{%
\chapter{Multipel regression}\label{multipel-regression}}

Som vi såg i vårt förra exempel så fann vi ett statistiskt signifikant samband mellan utomhustemperatur och polisanmäld brott samt att vår enkla regressionsmodell kunde förklara 46\% av variationen av polisanmäld brottslighet. Även om denna siffra är hög så kan vi tänka oss att det finns ytterligare faktorer som möjligtvis kan förklara varför brottslighet varierar från dag till dag. En sådan faktor är om det regnar utomhus. Därför kommer vi att inkludera en kontrollvariabel til vår enkla, bivariata regression. På så sätt blir det en multipel regression. Variabeln regn är en så kallad dummyvariabel vilket kan anta två värden: 1 för dagar det regnar och 0 för dagar då det ej regnar. Vår regressionsmodell kommer då se ut som följande:

\(Y_{brott} = b_0 + b_{celcius} + b_{regn}\)

Det vår modell säger är att vikan predicera polisanmälda brott utifrån vårt intercept/konstant (b0), utomhustemperaturen och förekomsten av regn.

Att utföra en multipel regression är snarlikt tillvägagångssättet för en enkel regression. Gör följande:

Analyze \textgreater{} Regression \textgreater{} Linear

Precis som vid en enkel regression så drar vi in vår beroende variabel (polisanmäld brottslighet) till Dependent-fältet. Gör därefter precis som vid den enkla regressionen och dra in den oberoende variaben temperatur till Independents-fältet.

Tryck därefter på knappen Next. Mata därefter in temperaturvaribeln på nytt och sedan vår kontrollvariabel regn. Tryck därefter på OK.

Vi hade lika gärna kunnat vid första steget mata in båda våra oberoende variabler och inte i två separata ``block''. Fördelen med att använda blockfunktione är att vi i SPSS först får ut vår enkla regression och därefter i samma ruta den multipla regressionen. Fördelen med detta tillvägagångssätt är att vi kan se hur effekten för vår oberoende variabel av intresse förändras från den enkla regressionen till den multipla regressionen.

Längst ut till vänster i samtliga rutor kan i se indikatorer på om de olika värdena avser modell 1 (den enkla regressionen) eller modell 2 (den multipla regressionen). Värdena i modell 1 är identiska som i föregående exemplet när vi utförde en enkel regression.

Vi börjar med att se hur vårt R Square-värde förändras. Eftersom vi har fler än en oberoende variabel så använder vi oss av värdet från Adjusted R Square vilket är 0.532. Det betyder att våra oberoende variabler kan förklara 53.2\% av variationen i vår beroende variabel vilket är en ökning med cirka 7 procentenheter jämfört med vad den enkla regrssionen klarade av att förklara.

I Coefficients-rutan och modell 2 kan vi se att konstanthållet för förekomsten av regn så ökar antalet brott med 43.081 för varje enhets ökning av temperaturvariabeln. Under Sig. kan vi se att p-värdet är under 0.001 vilket beyder att sambandet mellan temperatur och brott är statistiskt signifikant även efter inkluderingen av den oberoende variabeln regn.

Vidare kan vi se att koffecienten för den oberoend variabeln regn är -109.373. Det betyder att konstanthållet för den övriga oberoende variabler så begås det 109 färre brott under de dagar det regnar jämfört med regnfria dagar.

Slutligen ska vi försöka predicera antalet brott det anmäls en dag då det är 22 celcius och regnar

\(Y_{brott} = b_0 + b_{celcius} + b_{regn}\)

\(Y_{brott} = -693,137 + (43,081*25) + (-109,373*1)\)

\(Y_{brott} = 274.515\)

Vi predicerar alltså att det i genomsnitt kommer polisanmälas 374 brott de dagar då utomhustemperaturen är 22 grader och regnar.

\hypertarget{logistisk-regression}{%
\chapter{Logistisk regression}\label{logistisk-regression}}

När den oberoende variabel är numerisk och den beroende variabel dikotom och således enbart kan anta två värden har man traditionellt använt sig av analysmetoden logistisk regression.
Här nedan följer först instruktioner hur man utför en enkel logistisk regression och som sedan följs upp av instruktioner hur man genomför en multipel logistisk regression.

\hypertarget{enkel-logistisk-regression}{%
\section{Enkel logistisk regression}\label{enkel-logistisk-regression}}

I detta exempel kommer att använda oss av Pathways to desistance-datasetet. Den frågeställning vi ställer är följande: Finns det ett samband mellan att ha skjutit en annan person och att ha en pappa som lagförts för ett brott?'' Ett annat sätt att formulera frågan är: Kan vi predicera huruvida en individ har skjutit en annan person utifrån information om faderns lagföringshistorik? Den beroende variabeln är dikotom och kan anta värdet 0 om individen ej har skjutit en annan person och 1 om personen tidigare har skjutit en annan person. Den oberoende variabeln antar värdet 0 om personens fader aldrig tidigare lagförts för ett brott och 1 om fadern har tidigare lagförts för ett brott. För att genomföra den enkla logistiska regressionen gör ni som följande:

Analyze \textgreater{} Regression \textgreater{} Binary Logistic

Därefter placerar ni den beroende variabeln - skjutit en annan person - i Dependent-fältet och den oberoende variabeln - lagförd fader - i Covariates-fältet.

Inte helt oväntat kommer ni få ut en rad olika rutor med en mängd information. Vi tar en i taget. Rutan Case Processing Summary ger er deskriptiv information om hur många individer som förekommer i analysen samt hur många som är missing. I rutan Dependent Variable Coding framgår det hur SPSS har valt att koda om den beroende variabeln så att den verkligen enbart antar värdena 0 och 1. Därefter kommer rad rutor under Block 0: Beginning Block. Block 0 är den logistiska modellen utan en oberoende variabel. Det vill säga en modell med enbart interceptet (och dess information om individen skjutit någon eller ej). Vi kan hoppa över hela Block 0 eftersom ingen information där är av intresse för oss.

Under Block 1: Method = Enter börjar det bli lite intressantare. Under Omnibus Tests of Model Coefficients och det tillhörande P-värdet för Model hittar ni information om vår angivna modell presterar bättre än en modell utan vår oberoende variabel. Det motsvarar informationen i ANOVA-rutang ällande f-färdet och p-värdet för regressionsmodellen när ni utför en regressionsanalys (se kapitlet om enkel och multipel regression). Vi kan i vårt fall se att vår modell är signifikant. Under Model Summary hittar ni information om hur pass bra er logistiska modell passar er data. Cox \& Snell R Square och Nagelkerke R Square är så kallade pseudo-R2 som ska efterlikna R2-värdet man får ut när man genomför en regressionanalys. Värdena för pseudo-R2 kan dock inte tolkas som estimat för förklarad varians. Enklast är istället att tänka att värdena kan gå från 0 till 1 och att höga värden indikerar att de oberoende variablerna är bättre på att predicera den beroende variabeln. I vårt fall har vi överlag ganska låga värden vilket betyder att även om den kan hjälpa oss predicera den beroende variabel så är dess prediktiva förmåga låg. -2 Log Likelihood är lite svårare att tolka men enkelt uttryck kan man säga att ju lägre värde desto bättre.

Under Classification Table hittar ni information om hur ofta (uttryckt i procent) som vår modell lyckades predicera rätt att en individ skjutit en annan person (alternativt ej) utifrån information om faders tidigare lagföring. Vår modell lyckades pricka rätt 87,9 \% av fallen. Även om denna siffra är hög så ska det dock noteras att vår modell lyckades inte predicera att någon av våra individer som skjutit en annan person trots att vi faktiskt har 162 individer som gjort det. Det går att förklara med kombinationen att det är förhållandevis få som skjutit en annan person och det faktu att vi har en oberoende variabel som har en låg prediktiv förmåga.

Nu kommer vi äntligen till den viktigaste rutan: Variables in the Equation. Här ser vi information gällande vår hypotesprövning så som vilken effekt vår oberoende variabel har på vår beroende samt om denna effekt är statistisk signifikant. När vi utför en logistisk regression uttrycks betakoefficienten i termer av så kallade logaritmerade oddskvoter (log odds ratios). Att tolka logaritmerade värden är minst sagt krångligt så därför brukar man ta exponenten av de logaritmerade värdena så att man får ut icke-logaritmerade betakofficienter. Dessa värden hittar ni under Exp(B). Enklaste sättet att tolka dessa koeffeicitenter är att värden som överstiger 1 visar positiva samband och värden som understiger 1 visar negativa samband. Ett värd på 1 visar inget samband alls. Eftersom vår oberoende variabeln har koefficienten 2.364 så har vi ett positivt förhållande: det finns ett samband mellan att ha en lagförd fader och att ha skjutit en annan person. Eftersom p-värdet är \textless{} .001 är detta samband statistiskt signifikant.

\textbf{Överkurs:} Även om icke-logaritmerade oddskvoter är enklare att tolka än logaritmerad så är estimaten fortfarande något krångliga att förstå. Vad betyder det exempelvis att vår oberoende variabel har värdet 2.364? Eller för den delen att konstanten är .095?

Konstanten är som vi alla vet de vars värde på de eller den oberoende är noll. Variabeln i vårt fall är de vars far ej lagförts för ett brott. Oddset för att ha skjutit en annan person är därför 0.095 för de som ej har en far som tidigare lagförts. Eftersom 0.095 är cirka en tiondel av 1 kan vi säga att det är nära tio gånger lägre odds att en individ med en icke-lagförd fader ej skjutit någon än ha gjort det. Vad säger då vår oberoende variabel med värdet 2.364? Jo, den beskriver att oddset att ha skjutit någon är 2.364 gånger högre för en individ vars fader är lagförd än en individ vars fader ej är lagförd.

Av pedagogiska skäl har vi som sagt valt en dikotom oberoende variabel men hur skulle man tolka koefficienten om den var kontinuerlig istället? Säg att den oberoende variabeln var faderns inkomst uttryckt i dollar per månad och att koefficienten fortfarande är 2.364. Koefficienten hade då tolkats som den genomsnittliga förändringen i odds per en enhets ökning. Så en individ vars far har en lön på 1000 dollar per månad har 2364 högre odds (2.364x1000) att ha skjutit någon än en individ fars far har 0 i inkomst (ja, jätteorealistiskt exempel men vi kör på detta av pedagogiska skäl).

\hypertarget{multipel-logistisk-regression}{%
\section{Multipel logistisk regression}\label{multipel-logistisk-regression}}

Nu är det dags att inkludera en ytterligare variabel till vår modell. Kan det kanske teoretiskt vara så att det finns en viss könsskillnad i den effekt som faderns lagföring har på risken att ha skjutit någon annan? Det kanske är så att pojkar löper högre risk eftersom det teoretiskt kan vara så att pojkar tar efter sin fader i högre utsträckning än flickor gör. För att justera/kontrollera för denna effekt inkluderar vi därför variabeln kön i vår modell. Vi gör i princip på samma sätt som tidigare:

Analyze \textgreater{} Regression \textgreater{} Binary Logistic

Vi börjar med att placerar vår beroende varibel - skjutit annan person - i Dependent-fältet. För att kunna se hur vår koefficient för lagförd fader förändras när vi går från en enkel till multipel logistisk regression så väljer vi att ha två modeller. Börja med att placera enbart den oberoende variabeln lagförd fader i Block 1-rutan. Tryck sedan på next och lägg på nytt den oberoende variabeln lagförd fader i rutan som nu dock heter Block 2. Placera därefter även in den obeoende variabeln kön i samma ruta. Tryck därefter på OK.

Precis som tidigare så får vi en uppsjö av rutor och denna gången även fler. Vi hoppar direkt ner till Block 1: Method = Enter (eftersom de föregående rutorna har vi samtliga redan gått igenom). Denna ruta är vår enkla logistiska regression och är således identisk med den som vi utförde här ovan. Alla värden ska vara identiska som i den enkla logistiska regressionen. Eftersom inget har förändrats här så går vi raskt över till Block 2: Method = Enter vilket är vår multipla logistiska regressions som innehåller modellen med våra två oberoende variabler.

\hypertarget{part-uxf6vrigt}{%
\part*{Övrigt}\label{part-uxf6vrigt}}
\addcontentsline{toc}{part}{Övrigt}

\hypertarget{bearbeta-tabeller-och-figurer}{%
\chapter{Bearbeta tabeller och figurer}\label{bearbeta-tabeller-och-figurer}}

För att öka förståelsen för den berättelse som en rapport eller uppsats vill förmedla gäller det att bland annat ha lättlästa (och snygga!) tabeller och figurer. Ibland kallas denna process för att göra tabeller och figurer publikationsredo. I detta kapitel kommer vi gå igenom hur man effektiviserar och gör SPSS-output estetiskt tillfredsställande. Att bearbeta figurer och tabeller medför oftast också att de komprimeras storleksmässigt vilket innebär att de inte tar lika stor plats i en text. Inte sällan tampas vi med ord- eller sidbegränsningar när vi skriver och tabeller som tar upp onöidigt stor plats kan leda till att vi inte kanske kan få med all den text som vi önskar.

\hypertarget{tabeller}{%
\section{Tabeller}\label{tabeller}}

I detta exempel ska vi effektivisera och snygga till en korstabell med tillhörand chi2-test. Notera att ni även kan följa denna guide även om ni gör en enkel frekvenstabeller (eller annan typ av tabell). I SPSS ser outputen ut som följande:

Notera för det första hur mycket av outputen som antingen behöver bearbetas eller som kan strykas. Det är exempelvis ord som är på englska, variabelnamn som innehåller tecken/ord som inte bör vara där, stora delar av chi2-testet som är överflöding information. Vi vill effektivisera outputen med målet att göra en tabell som ser ut som följande:

Notera hur vi har inte nog med skapat en snygg tabell även gjort den avsevärt mycket mer komprimerad vilket innebär att vi får plas med mer text. Chi2-tabellen som tidigare finns har helt försvunnit och dn information som är relevant finns nu under korstabellen.

För att återskpa den reviderade tabellen gör vi som följande:

\begin{enumerate}
\def\labelenumi{\arabic{enumi}.}
\item
  Innan vi ens skapar korstabellen klickar vi in oss på: Edit \textgreater{} Options och därefter fliken Pivot Tables.
\item
  Markera sedan APA\_SansSerif\_10pt. Klicka därefter på OK. Vi har nu angett att samtliga tabeller som vi skapar ska följa den mall vi precis angivit (notera att det inte finns något krav att just denna mall skall följas utan ni kan klicka er runt och använda er av andra mallar om ni önskar).
\end{enumerate}

\begin{enumerate}
\def\labelenumi{\arabic{enumi}.}
\setcounter{enumi}{2}
\item
  Skapa därefter den korstabell ni önskar.
\item
  När ni har fått fram tabellen i output-fönstret så dubbelklickar ni var som helst på korstabellen.
\item
  Ni ska nu ha fått fram Pivot Table-rutan. Det är här ni redigerar innehållet i er tabell.
\end{enumerate}

\begin{enumerate}
\def\labelenumi{\arabic{enumi}.}
\setcounter{enumi}{5}
\item
  Vi börjar med att skapa en kommentar under tabellen med information från chi2-test. Klicka på Insert \textgreater{} Caption. Skriv därefer in chi2-värdet (Pearson Chi-Square), frihetsgraderna (df) och därefter p-värdet (Asymptotic Significance (2-sided)) precis som vi gjorde i den färdiga tabellen här ovan.
\item
  Nu ska vi redigera variabelnamnen, titeln och annat så att tabellen ser prydlig ut. För att redigera rubrik och annan text dubbelklickar ni helt enkelt på delen av tabellen som ni vill textredigera. Ett textfält ska då dyka upp där ni skriver in den nya text ni önskar. Notera att SPSS kan vara lite envis ibland och vägra ge er textfältet. Dubbelklicka på nytt eller testa att klicka någon annan stans och sedan dubbelklicka på nytt på texten så ska SPSS till slut ge med sig.
\end{enumerate}

\begin{enumerate}
\def\labelenumi{\arabic{enumi}.}
\setcounter{enumi}{7}
\item
  När ni känner er klara så klicka File \textgreater{} Close. Nu ska er tabell ha uppdaterats utifrån era önskemål.
\item
  För att förflytta tabellen från output-fönstret till Word så högerklickar ni på tabellen, väljer Copy As \textgreater{} Image. Tabellen är nu kopierad som bild vilket innebär att ni nu kan klistra in den i Word.
\end{enumerate}

Nu är nu klara och har en avsevärt prydligare och komprimerad tabell!

\hypertarget{figurer}{%
\section{Figurer}\label{figurer}}

Figurera i SPSS är till stora delar redan publikationsredo och det som ibland kan vara lämpligt att förändra är kanske mindre kosmetiska saker så som att översätta engelska ord eller se till att vissa ord börjar på stor bokstav. Målen med denna guide är att gå från följande figur:

Till följande:

Förändringarna vi har gjort är inte lika stora som förändringarna för frekvenstabellen här ovan. Inte heller är de förändringar som gjorts nödvändiga utan vissa av åtgärderna är mest gjorda för att visa er var i SPSS som man kan göra förändringar av figurer. Det som är förändrat är att de horisontella linjerna (grid lines) är borttagna, färgen på staplarna är förändrade, stor bokstav på variabelvärdena, tagit bort rubriken och översatt engelskt ord. Det finns även andra saker man kan tänkas göra för att förbättra figuren men dessa få åtgärder är en bra start och visar vilka fönster som finns att tillgå för att förändra en figur.

Börja med att skapa en figur via Graphs \textgreater{} Chart builder eller Charts-knappen när ni skapar en frekvenstabell. Därefter dubbelklickar ni var som helst på figuren alternativt högerklickar och trycker på Edit. Chart editor-fönstret ska nu öppnas upp. Det är härifrån ni gör samtliga era förändringar på figuren.

Först ut tar vi bort rubriken. Högerklicka på rubriken och tryck därefter på Delete. Nu är rubriken borta. Om ni iställt vill redigera rubriken så enkelklickar ni på rubriken. Vänta sedan någon sekund och enkelklicka igen på rubriken. Nu kan ni redigera den.

Näst på tur är att översätta rubiken för y-axeln, från engelska till svenska. Enkelklicka på texten, vänta en sekund och enkelklicka sedan igen. Nu kan ni redigera texten till svenska.

För att redigera variabelnvärdena på x-axeln så att de alla börjar med stor bokstav så gör ni på samma sätt som här ovan. Enkelklicka på en av värdena, vänta en sekund och sedan enkelklicka på nytt. Nu kan ni redigrera texten. Gör så på samtliga variabelvärden.

För att ta bort de horisontella linjerna (grid lines) trycker ni på Options \textgreater{} Hide Grid Lines.

Slutligen ska vi ändra färgen på staplarna. Dubbelklicka på någon av staplarna. Tryck sedan på fliken Fill \& Border. Välj därefter vilken färg ni vill ha på staplarna (under Fill). Tryck därefter på Apply.

När ni känner er klara med era förändringar så klickar ni på File \textgreater{} Close. Nu ska er befintliga figur har uppdaterats efter era önskemål.

\hypertarget{presentation-av-dataset}{%
\chapter{Presentation av dataset}\label{presentation-av-dataset}}

\hypertarget{ntu-2017-2021}{%
\section{NTU 2017-2021}\label{ntu-2017-2021}}

Datasetet NTU 2017-2021 M2.SAV innehåller fem årgångar av NTU och består sammanlagt av 375 590 respondenter. Samtliga individer är anonymiserade vilket innebär att det ej går att härleda vem som förekommer i datamaterialet. Vilka variabler som finns i datamaterialet och hur varje variabel är kodad går att se i kodboken för datamaterialet som ni finner i Athena. Detta dokument är viktigt att nyttja i samband med att ni använder datasetet då det inte alltid framgår vad varje variabel betyder och vad dess värden innebär.

Närmare information om urval, datainsamling, frågekonstruktion, kodning och annat relevant återfinns i den
tekniska rapporten för NTU 2021 som ni finner här.

\hypertarget{ntu-2013-2015}{%
\section{NTU 2013-2015}\label{ntu-2013-2015}}

Datafilen ''NTU 2013-15 M2.sav'' är en SPSS fil som rymmer tre hela årgångar av NTU,
sammanlagt 37 118 observationer (personer som besvarat frågorna). Det går inte att identifiera
vilket av de tre åren som respektive person ingått i undersökningen, varför datamaterialet
behandlas som en tvärsnittsundersökning (NTU 2013-15).

Närmare information om urval, datainsamling, frågekonstruktion, kodning mm återfinns i den
tekniska rapporten för NTU 2015 som ni finner här.

För att arbeta med NTU datamaterialet behöver ni hjälp av information från den tekniska
rapporten, exempelvis ser ni där hur frågor och svarsalternativ är utformade.
Utifrån variabelnamnen i datafilen går det att identifiera frågorna i frågeformuläret.

Exempel:

I datafilen finns en variabel som heter C6, av variabelns label (etikett) framgår att den rör
cykelbrott: ''Cykelstöld\_CY\_C6''. Att frågan rör cykelstöld är tydligt, men vilken av alla frågor
som rör cykelstöld? Genom att frågenumret i frågeformuläret anges i variabeländelsen, i
exemplet ''C6'', kan vi koppla till frågeformuläret i den tekniska rapporten (Brå 2016, bilaga 1,
sid 5):

Ni har enbart tillgång till vissa frågor i NTU -- bakgrundsfrågor om t.ex. ålder och kön samt
frågor om utsatthet för brott, oro för brott och förtroende för rättsväsendet.

\hypertarget{skolundersuxf6kning-2005}{%
\section{Skolundersökning 2005}\label{skolundersuxf6kning-2005}}

Datafilen ''Skol05.sav'' är en SPSS fil med data från Brottsförebyggande rådets
skolundersökning om brott från 2005 (SUB2005). Undersökningen rymmer ursprungligen 7449
observationer (deltagande elever), ni har dock ett slumpurval av hälften av de svarande (3724
elever). Ni har tillgång till de flesta frågor som ingick i studien, tex om egen brottslighet och
utsatthet för brott (totalt rör det sig om cirka 190 variabler).

Närmare information om urval, bortfall, datainsamling, frågekonstruktion, kodning mm återfinns
i den tekniska rapporten för SUB 2005 som ni finner här.

För att arbeta med skolundersökningen behöver ni hjälp av information från den tekniska
rapporten, exempelvis ser ni där hur frågor och svarsalternativ är utformade. Utifrån variabelnamn och label i datafilen går det enkelt att identifiera frågorna, för deras exakta
lydelse får ni dock gå till den tekniska rapporten.

\hypertarget{glass-och-brott}{%
\section{Glass och brott}\label{glass-och-brott}}

Inte sällan inom samhällsvetenskapen brukar det faktum att korrelation inte nödvändigtvis innebär att det finns ett orsakssamband exemplefieras med hjälp av sambandet mellan glassförsäljning och brott. Ju fler glassar som sälj under en dag desto fler brott brukar i regel anmälas. Detta samband är dock ett skensamband efters det finns en bakomliggand variabel som orsakar både fler sålda glassar och fler polisanmälda brott: temperatur.

Ett fiktivt (!) datasetet (ice\_cream.sav) har skapats för att vidare undersöka detta samband. Datasetet innehåller 60 dagar och som ska representera två sommarmånader. Varje dag innehåller information om hur många brott som polisanmäts, hur många celcius som uppmättes under dagen och om det förekom regn under dagen.

\hypertarget{pathways-to-desistance}{%
\section{Pathways to desistance}\label{pathways-to-desistance}}

The Pathways to Desistance-studien (PATHWAYS\_01Baseline.sav) är en longitudinell studie som utfördes på flera platser i USA. Materialet består av cirka 1 300 tonårsbrottslingar under den period i deras liv när de övergår från tonåren till tidig vuxen ålder. De inskrivna ungdomarna var minst 14 år och under 18 år när de begick brott och befanns skyldiga till ett allvarligt brott (främst grova brott, med några få undantag för vissa förseelser egendomsbrott, sexuella övergrepp eller vapenbrott). Varje studiedeltagare följdes upp under en period av tre år efter inskrivningen. Materialet ger en heltäckande bild av livsförändringar inom ett brett spektrum av områden under loppet av tiden de observeras.

Datamaterialet består av nära 1 000 olika variabler som berör bland annat information om egen brottslighet, skolprestation, familjeförhållanden, umgängeskrets, mental hälsa, rutinaktiviteter. Övergripande information om vilka variabler som förekommer finner ni här. För mer detaljerad information om respektive variabel finner ni här. Här kan ni även söka efter de variabler som ni finner i datasetet och få en vidare beskrivning samt vilka värden varje variabel har.

  \bibliography{book.bib,packages.bib}

\end{document}
